\documentclass{article}
\usepackage{graphicx} % Required for inserting images

\usepackage{amsmath, amssymb}
\usepackage{hyperref}
\usepackage{geometry}
\geometry{margin=1in}

\title{Personal Thesis}
\author{율표 주기}
\date{Jun 2025}

\begin{document}

\maketitle

I believe that \textbf{algebraic formulas} are notional concepts that describe the \textbf{properties of the concept of number}.\\
The concept of number is, to me, a \textbf{measurable and conserved concept}, which can be judged through discrete logical statements.\\
An algebraic formula contains \textbf{logical flows}, and the following is a list of formulas derived from those logical flows.

\bigskip
\hrule
\bigskip

\section*{Representing Logical Operations via Arithmetic Operations}

First, define some functions to be used:

\begin{itemize}
    \item $\mathrm{Solve}(F) \coloneqq \{\bar{x} \mid F(\bar{x}) = 0 \}$
    \item $\mathrm{boolf}(S)(x) \coloneqq (x \in S)$
    \item $\mathrm{bool}(x) \coloneqq (x \neq 0)$
\end{itemize}

\bigskip
\hrule
\bigskip

\subsection*{1. 3-variable: $x, y \to z$}

Logical value assignments:

\begin{align*}
0000:& \quad z = 0 \\
0001:& \quad z = 1 + xy - (x + y) \\
0010:& \quad z = y - xy \\
0011:& \quad z = 1 - x \\
0100:& \quad z = x - xy \\
0101:& \quad z = 1 - y \\
0110:& \quad z = x + y - 2xy \\
0111:& \quad z = 1 - xy \\
1000:& \quad z = xy \\
1001:& \quad z = 1 + 2xy - (x + y) \\
1010:& \quad z = y \\
1011:& \quad z = 1 + xy - x \\
1100:& \quad z = x \\
1101:& \quad z = 1 + xy - y \\
1110:& \quad z = x + y - xy \\
1111:& \quad z = 1
\end{align*}

\bigskip

\subsection*{2. 2-variable: $x \to y$}

Logical value assignments:

\begin{align*}
00:& \quad y = 0 \\
01:& \quad y = 1 - x \\
10:& \quad y = x \\
11:& \quad y = 1
\end{align*}

\bigskip

\subsection*{3. 0-variable Linear Mapping $xe$}

\begin{itemize}
    \item Logical assignment $0$ maps to $0$
    \item Logical assignment $1$ maps to $1$
\end{itemize}

\bigskip
\hrule
\bigskip

\section*{Model Theory and Equations}

To judge \textbf{equality of values}, define:

\[
\delta_i(x) \coloneqq \lim_{n \to (x - i)^+} 0^n
\]

By interpreted proposition $P(x) = 0$ into the logical truth value which is $\{0, 1\}$,\\
then the composed function $\delta_0 \circ P$ computes the truth of the proposition $P(x) = 0$.

\subsection*{Key Insight: Logical Meaning of an Equation: The "Existential ($\exists$) Condition"}

About function $P$, "the equation which is introduced by function $P$" is a solution (root) to the equation $P(x) = 0$.\\

Thus, "the equation which is introduced by function $P$" is logically equivalent to the predicate logic statement:

\[
(\exists x)(P(x) = 0)
\]

Which can also be written using composed functions:

\[
(\exists x)((\mathrm{bool} \circ \delta_0 \circ P)(x))
\]

\bigskip
\hrule
\bigskip

\section*{Background: Logical Consequence}

If an assignment $\bar{x}$ satisfies a proposition $p$, we write $\bar{x} \vDash p$, and call this a \textbf{satisfying relationship}.

\begin{itemize}
    \item For a proposition $p$, define its \textbf{model set}:
    \[
    \mathrm{Mod}(p) \coloneqq \{\bar{x} \mid \bar{x} \vDash p \}
    \]
    \item For a set of propositions $\Phi = \{p_1, p_2, \ldots, p_n \}$:
    \[
    \mathrm{Mod}(\Phi) \coloneqq \bigcap \mathrm{Mod}(p)
    \]
    \item The \textbf{logical consequence} relation $A \vDash B$ means:
    \[
    \mathrm{Mod}(A) \subseteq \mathrm{Mod}(B)
    \]
\end{itemize}

\bigskip

\subsection*{Key Insight: Algebraic Expression of Logical Consequence}

The logical consequence relation $A \vDash B$ can be expressed arithmetically using the divisibility operator:

\[
\mathrm{bool}^{-1}(A) \mid \mathrm{bool}^{-1}(B)
\]

Note: Since $x \equiv y \pmod{m}$ means $m \mid (x - y)$,\\
we can interpret $f \mid g$ as $g \equiv 0 \pmod{f}$.

Thus, if $g \bmod f = 0$, then $g$ is a logical consequence of $f$, and this itself ($g \bmod f = 0$) is an equation.

\bigskip

\subsection*{Key Insight: Logical Meaning of an Identity ($\forall$ Condition)}

If $f(x) = 0$ is an identity (always true), then $f(x) \neq 0$ is \textbf{inconsistent}.\\

Thus, $(\exists x)(\mathrm{bool}((1 - \delta_0 \circ f)(x)))$ is always false, and\\
$\neg (\exists x)(\mathrm{bool}((1 - \delta_0 \circ f)(x)))$ is always true.\\

Hence, the identity $f$ can be expressed as:

\[
\neg (\exists x)(\mathrm{bool}((1 - \delta_0 \circ f)(x)))
\]

\bigskip
\hrule
\bigskip

\section*{Applying "Logical Flows in Algebraic Formulas" to Arithmetize Logic}

Covering:

\begin{itemize}
    \item Equations (including inconsistent, identity, and standard types)
    \item Functions (predicates, operators)
    \item Propositional logical connectives
    \item Logical consequence in model theory
    \item Equations as existential statements
    \item Identity equations as universal statements
\end{itemize}

\bigskip

\textbf{Examples}:

\begin{align*}
\mathrm{Solve}(\lambda x . 1) &= \mathbb{R} \quad \text{(the universal set, all real numbers)} \\
\mathrm{Solve}(\lambda x . 0) &= \varnothing \quad \text{(empty set)} \\
\mathrm{Solve}(\lambda x . a x^2 + b x + c) &= \left\{ \frac{-b \pm \sqrt{b^2 - 4ac}}{2} \right\}
\end{align*}

Also:

\[
(\mathrm{bool}^{-1} \circ \mathrm{boolf})(S) = 1_S
\]

\[
(\mathrm{bool}^{-1} \circ \mathrm{boolf} \circ \mathrm{Solve})(x) = 1_{\mathrm{solve}(x)}
\]

Therefore, by using the function composition $\mathrm{bool}^{-1} \circ \mathrm{boolf} \circ \mathrm{Solve}$,\\
one can perform \textbf{predicate logic using arithmetic}, without having to define natural-language predicates. \dots (1)

Further, for any natural-language predicate $P$, one can \textbf{port} it into arithmetic using $\mathrm{bool}^{-1} \circ P$. \dots (2)

Hence, by conclusions (1) and (2), \textbf{logical operations can be expressed via algebraic operations}.

\bigskip

\subsection*{Note on Conclusion (1)}

This works because for any algebraic operation $f$,\\
the function $\mathrm{Solve}(f)$ already serves the role of $\mathrm{boolf}^{-1} \circ \mathrm{bool}^{-1}$.\\

Therefore, unless external set theory is used,\\
$\mathrm{Solve}$-defined sets are \textbf{set-theoretically implicit}, not external.

\bigskip
\hrule
\bigskip

\section*{Bonus 1}

This logical framework can be smoothly extended to:

\begin{itemize}
    \item \textbf{Fuzzy set logic} (membership degree, like probability)
    \item \textbf{Multiset logic} (multiplicity degree)
    \item \textbf{Fuzzy multiset logic} (membership + multiplicity)
\end{itemize}

\bigskip

\section*{Bonus 2}

By using \textbf{adjacency matrices} from linear algebra and assigning objects to node sequences,\\
this framework can incorporate \textbf{graphs}.

This allows the use of:

\begin{itemize}
    \item Category theory
    \item Fuzzy/multiset/general sets (via subset notation)
    \item Graph representations dependent on natural-language predicates and node structures
\end{itemize}

\textbf{Caveat}: When extended to graphs, the underlying structure becomes \textbf{linear algebra},\\
thus using \textbf{tensors as algebraic entities}, which \textbf{goes beyond high school mathematics}.\\

That would deviate from the \textbf{secondary goal}:

\begin{quote}
\textit{``A complete mathematical logic system expressible via pure high school algebra.''}
\end{quote}

This is why I’m trying to resolve things using only \textbf{Euclidean space}.

\bigskip
\hrule
\bigskip

\section*{Final Thought}

I see:

\begin{itemize}
    \item Numbers as \textbf{measurable conserved concepts}
    \item Algebraic formulas as \textbf{statements about numbers}
    \item Logic embedded in those formulas as the \textbf{bridge between mathematical logic and symbolic logic}
    \item Mathematical logic as the \textbf{descriptive system of the logical context implied by numbers}
\end{itemize}

\end{document}
